\section{Atticus Wang: The $\A^1$-Euler class (Draft)}

\subsection{Preliminaries}

\subsubsection{The categorical Euler characteristic}

The Euler characteristic of a space $X$ homotopy equivalent to a finite CW complex is defined as
\[\chi(X):=\sum_d(-1)^d \dim H^d(X;k)\]
where $k$ is any field. It is easy to show that the definition does not depend on the field. Moreover, given a cell decomposition of $X$ with $c_d$ cells in dimension $d$, we have
\[\chi(X)=\sum_d (-1)^d c_d.\]


One can define the Euler characteristic of any perfect complex over a field $k$ as a similar alternating sum. To categorify this construction, we recognize that the dimension of a finite-dimensional vector space $V$ can be categorified as the composition
\[k \xto{\coev} V \otimes V^\vee \xto{\tau} V^\vee \otimes V \xto{\ev} k\]
where the first map is the coevaluation mapping $1$ to $\id\in \End V\simeq V\otimes V^\vee$, the second map transposes the two factors, and the third map is the evaluation $w\otimes v\mapsto w(v)$ (also known as the trace of a matrix). 

\begin{definition}

An object $X$ in a symmetric monoidal category is \emph{dualizable}\footnote{They are called \emph{strongly dualizable} in the original paper by Dold--Puppe \cite{DoldPuppe}. Following standard abuse of terminology, we just call them dualizable.} if there exists another object $X^\vee$, and maps 
\[\coev: \mathbbl{1}\to X\otimes X^\vee, \quad \ev: X^\vee\otimes X\to \mathbbl{1}, \]
such that the compositions
\[X\xto{\coev\otimes \id} X\otimes X^\vee\otimes X \xto{\id\otimes \ev} X \]
\[X^\vee\xto{\id\otimes \coev} X^\vee\otimes X \otimes X^\vee \xto{\ev\otimes \id} X^\vee\]
are identity.

\end{definition}

It is clear then that $X^\vee$ represents the functor $A\mapsto \Hom(A\otimes X, \mathbbl{1})$, and the data $(X^\vee, \coev, \ev)$ will be unique up to unique isomorphism, if it exists. Also, $X^\vee$ is dualizable with dual $X$, and a map $X\to Y$ gives rise to a dual map $Y^\vee \to X^\vee$.

\begin{definition}

Given a dualizable object $X$, define its Euler characteristic to be the composition
\[\mathbbl{1} \xto{\coev} X\otimes X^\vee \xto{\tau} X^\vee\otimes X \xto{\ev} \mathbbl{1}\]
landing in the associative algebra $\End(\mathbbl{1})$. 
\end{definition}

The alternating signs are implemented by the transposition $\tau$: if we were in the category of chain complexes, for example, then $\tau$ would follow the Koszul sign rule. Dualizable objects in the derived category $\sD(A)$ of a ring $A$ are precisely the perfect complexes, and it is clear that the above definition generalizes the usual Euler characteristic.


\subsubsection{The Grothendieck--Witt ring}

Let $k$ be a field, $\characteristic k \neq 2$. We recall some classic facts about symmetric bilinear forms below.


\begin{theorem}
Any symmetric bilinear form can be diagonalized (i.e.~is isomorphic to a diagonal one).	
\end{theorem}

We can thus denote a symmetric bilinear form by the diagonal entries $\langle a_1,\dots,a_n\rangle$, which are of course not uniquely determined. 

Recall that a \emph{hyperbolic plane} is the vector space $k^2$ with bilinear form $\langle 1, -1\rangle$.

\begin{theorem}[Witt decomposition theorem]
Let $V$ be a finite-dimensional vector space over $k$ with a symmetric bilinear form $B$. Then we can block-diagonalize $B$ as $V\simeq V_0 \oplus H\oplus W$, where $B$ vanishes on $V_0$, $H$ is a direct sum of hyperbolic planes, and $W$ is anisotropic (meaning $B(w,w)\neq 0$ for any $w\in W$).
\end{theorem}

\begin{theorem}[Witt cancellation theorem]
\label{theorem:Witt-cancel}
If $B_1,B_2,B_3$ are finite-dimensional symmetric bilinear forms, then $B_1\oplus B_2\simeq B_1\oplus B_3$ implies $B_2\simeq B_3$.
\end{theorem}


Consider $M(k)$, the semiring of isomorphism classes of finite-dimensional nondegenerate symmetric bilinear forms over $k$, with addition given by direct sum and multiplication given by tensor product. Its group completion is called the Grothendieck--Witt ring $\GW(k)$. By \cref{theorem:Witt-cancel}, the map $M(k)\to \GW(k)$ is injective.

\begin{exm}
	$\GW(\CC)=\ZZ$ given by (virtual) dimension, $\GW(\RR)=\ZZ\oplus \ZZ$ given by dimension and signature (Sylvester's law of inertia).
\end{exm}

\begin{exr}
	The additive subgroup generated by the hyperbolic plane is an ideal in $\GW(k)$. 
\end{exr}

Define the \emph{Witt ring}\footnote{Unrelated to Witt vectors.} $\W(k)$ as the quotient of $\GW(k)$ by the above ideal. Having gotten rid of degenerate and hyperbolic parts, the Witt ring counts the possible anisotropic bilinear forms over $k$. Denote $\I(k)=\ker(\dim: \W(k)\to \ZZ/2)$.

\begin{proposition}[Milnor's conjecture for $n=1$]
The map $\det: \I(k)/\I(k)^2 \to k^\times/(k^\times)^2$ is an isomorphism.
\end{proposition} 

\begin{proposition}
	$\GW(k)$ is generated by $\langle a\rangle$ for $a\in k^\times$, with relations
	\begin{enumerate}
		\item $\langle a \rangle = \langle ab^2\rangle$
		\item $\langle a \rangle\langle b\rangle = \langle ab\rangle$
		\item $\langle a \rangle + \langle b \rangle = \langle a+b\rangle + \langle ab(a+b)\rangle$ for $a+b\neq 0$
		\item $\langle a\rangle + \langle -a\rangle = \langle 1\rangle + \langle -1 \rangle$.
	\end{enumerate}
\end{proposition}


\subsubsection{The six functor formalism}

For a scheme $X$ quasi-projective over a fixed field $k$ (or in general a base scheme), recall that we have the presentably symmetric monoidal stable $\infty$-category $\SH_X$, given by first taking the presheaf category $\PSh(\Sm_X)$ on smooth $X$-schemes valued in pointed spaces, then forming $\Spc_X$ as the universal localization of $\PSh(\Sm_X)$ satisfying Nisnevich descent and $\A^1$-invariance, then finally inverting the $\PP^1$-suspension endofunctor on $\Spc_X$. 

Given a morphism $X\to Y$ (over the base field $k$, as implicitly assumed from now on), we have functors $f_*,f_!:\SH_X\to \SH_Y$ and $f^*,f^!:\SH_Y\to \SH_X$ satisfying:
\begin{enumerate}
\item There are adjunctions $f^* \dashv f_*$ and $f_! \dashv f^!$;
\item There is a natural transformation $f_!\to f_*$. When $f$ is proper, this is an isomorphism;
\item When $f$ is smooth, there is a functor $f_\sharp: \SH_X\to \SH_Y$ left adjoint to $f^*$.
\item When $f$ is an open immersion, the pairs $(f_\sharp, f^*)$ and $(f_!,f^!)$ are isomorphic.
\end{enumerate}


\begin{exm}[localization triangle]
	Suppose $X$ is quasi-projective over $k$, let $j:U\into X$ be an open immersion, and $i:Z=X\backslash U\into X$ the reduced closed subscheme. Then we have a cofiber sequence
	\[j_!j^! \to \id \to i_*i^*\] 
	of endofunctors of $\SH_X$. Applying this to $\Sigma^\infty_+ X$, we get
	\[j_\sharp\Sigma^\infty_+ U \to \Sigma^\infty_+ X \to i_*\Sigma^\infty_+ Z,\]
	in other words $i_*\Sigma^\infty_+ Z \simeq \Sigma^\infty(X/U)$ where the latter quotient is the cofiber in $\Spc_X$.
\end{exm}

\begin{exm}[Thom space]
	Let $p:V\to Y$ be a vector bundle, and $s:Y\to V$ the zero section. Then by the above we have $p_\sharp s_*\Sigma^\infty_+ Y \simeq \Sigma^\infty\Th_Y (V)$ is the classical Thom construction. Define $\Sigma^V = p_\sharp s_* :\SH_Y\to \SH_Y$, then $\Sigma^VX=X\otimes \Sigma^\infty\Th_Y(V)$, and it is an equivalence with inverse $\Sigma^{-V}=s^!p^*$.
\end{exm}

\begin{exm}[Atiyah duality]
Let $f:X\to Y$ be smooth, and let $T\to X$ be the total space of the relative tangent bundle $T_{X/Y}$. Then Atiyah duality can be formulated as natural isomorphisms $f_! \simeq f_\sharp \Sigma^{-T}$, $f^!\simeq \Sigma^T f^*$.
\end{exm}

\begin{exm}[Borel--Moore homology]
Let $\pi:X\to Y$, and $Z\in \Spc_X$. Then the Borel--Moore pushforward of $Z$ is given by $\pi_!\Sigma^\infty Z\in \SH_Y$. In particular denote $X_{\BM}:= \pi_!\Sigma^\infty_+ X$. It turns out this is contravariant: suppose $f:X_1\to X_2$ is a proper map of quasi-projective $Y$-schemes, with structure maps $\pi_1,\pi_2$, then applying $(\pi_2)_!$ to the map $\Sigma^\infty_+X_2\to f_*\Sigma_+^\infty X_1$ gives a map $(X_2)_{\BM}\to (X_1)_{\BM}$.
\end{exm}




\subsection{Euler characteristics}

\subsubsection{Setup}

We will be mostly concerned with the stable motivic homotopy category $\SH_k$ over a field $k$. They are symmetric monoidal $\infty$-categories, and the tensor unit is denoted $\SS_k$, the analogue of the sphere spectrum $\SS$ in stable homotopy theory. 


\begin{theorem}[Morel]
	For $k$ a perfect field, $\End_{\SH_k}(\SS_k)\simeq \GW(k)$.
\end{theorem}

\begin{theorem}[]
	Let $\pi:X\to k$ be smooth and projective, then $\pi_\sharp \Sigma^\infty_+X\in \SH_k$ is dualizable. 
\end{theorem}


This is analogous to the classical Atiyah duality, which says that for a compact manifold $M$, the spectra $\Sigma^\infty_+M$ and $\Sigma^{-T(M)}\Sigma^\infty_+M$ are Spanier--Whitehead dual in the stable homotopy category. In our language, $\pi_!=\pi_\sharp\Sigma^{-T}$, so the claim is that $\pi_\sharp \Sigma^\infty_+X$ and $\pi_! \Sigma^\infty_+X$ are dual to each other. The construction of coev and ev are mostly formal.

\subsubsection{Serre's formula}

The following formula, proposed by Serre and proved by Levine--Raksit \cite{}, lets us compute the Euler characteristic in many practical cases. Recall that 

\begin{theorem}
	Su
\end{theorem}



\subsubsection{First computations}

Assume for the rest of the article that $k$ is perfect.

\begin{proposition}
	$\chi(\SS^{a,b}_k) = (-1)^a\langle -1\rangle^b$.
\end{proposition}



\subsection{Euler classes}

In this section we will sketch the proof of formula. To do this, we will identify the Euler characteristic as the integral of a certain Euler class (as in classical Gauss--Bonnet), and then identify the abstractly defined Euler class concretely. 

\subsubsection{Classical definition}

The Euler class is an invariant living in $H^k(X;\ZZ)$ for a rank $k$ oriented real vector bundle over $X$. Let us briefly recall its definition and basic properties.

\begin{definition}
	Let $V$ be a rank $k$ oriented real vector bundle over a paracompact topological space $X$, and let $s:X\to V$ be the zero section. Let $\tau$ be the Thom class viewed as an element of $H^k(V;\ZZ)$. Then we define the Euler class $e(V) = s^*\tau$.
\end{definition}



\begin{rem}
	The Euler class has the following geometric meaning when $X$ is a compact smooth manifold: take a generic section $t:X\to V$ which intersects the zero section $s$ transversely at some submanifold $Z$. Then the Poincar\'e dual of $[Z]\in H^{\dim X - k}(X)$ is the Euler class of $V$. In particular, if $\dim X = k$, then the Euler class is a signed count of the zeros of the section.
\end{rem}


\begin{rem}
Because oriented rank $k$ real vector bundles are the same as $\SO(k)$-torsors, they are classified by mapping into $B\SO(k)$, the double cover of $B\O(k)=\Gr(k,\RR^\infty)$. So the existence of Euler classes are explained by classes in the degree-$k$ cohomology of $B\SO(k)$. When $k$ is odd this class is 2-torsion; when $k$ is even this class is a polynomial generator which squares to the top Pontryagin class.
\end{rem}


\begin{theorem}[Chern--Gauss--Bonnet]
Let $M$ be a smooth compact oriented $n$-manifold, then $\chi(M) = \langle e(M), [M]\rangle$.
\end{theorem}

\begin{exm}[Lines on a cubic surface]

Let 
	
\end{exm}

\subsubsection{The motivic Euler class}

First we need to define the Thom class, which should live in cohomology of the total space $V$. We should replace $H\ZZ$ by any multiplicative cohomology theory of schemes which admit Thom classes for orientable vector bundles. Cohomology theories are represented by ring spectra, so let $E\in \CAlg(\SH(k))$ be $\SL$-oriented. In homotopy theory, a complex orientation gives every complex vector bundle an Euler class. Similarly:

\begin{definition}
	Let $E\in \CAlg(\SH_k)$. An \emph{$\SL$-orientation} of $E$ is the following data: to every vector bundle $p:V\to X$, where $X\in \Sm_k$, together with an isomorphism $\rho:\det V \to \sO_X$, we associate a class $\tau_{V,\rho} \in E^{2r,r}(\Th(V))$ where $r=\rk V$. They are required to satisfy:
	
	\begin{itemize}
		\item Naturality
		\item Products
		\item Normalization
	\end{itemize}
\end{definition}

As an exercise in definition, we prove the analogue of the Thom isomorphism theorem:

\begin{proposition}

There is an isomorphism $E^{*,*}(X)\to E^{*+2r, *+r}(\Th(V))$ mapping $x\mapsto p^*x\smile \tau_{V,\rho}$.
	
\end{proposition}


\begin{proof}
	When $V$ is a line bundle, by the normalization axiom this map is just $\PP^1$-suspension which is an isomorphism. By the product axiom this map is also an isomorphism when $V$ is a direct sum of line bundles. In general, one could cover $X$ by opens on which $V$ splits, and use the Mayer--Vietoris spectral sequence. Since the spectral sequences on both sides agree on $E_2$-page, the groups they converge to also agree.
\end{proof}

\begin{definition}
	Let $X,Y\in \Sm(k)$, $f:X\to Y$ proper, and $E$ a $\SL$-oriented spectrum. We have a pushforward map on $E$-cohomology.
\end{definition}


Suppose now that $X$ is a smooth projective variety over a field $k$, and $p:V\to X$ a vector bundle. Let $s:X\to V$ be the zero section. 




\subsection{Computational techniques}



\end{document}












